% ------------------------------------------------------------------------
% USTAWIENIA
% ------------------------------------------------------------------------

% ------------------------------------------------------------------------
%   Kropki po numerach sekcji, podsekcji, itd.
%   Np. 1.2. Tytuł podrozdziału
% ------------------------------------------------------------------------
\makeatletter
    \def\numberline#1{\hb@xt@\@tempdima{#1.\hfil}}                      %kropki w spisie treści
    \renewcommand*\@seccntformat[1]{\csname the#1\endcsname.\enspace}   %kropki w treści dokumentu
\makeatother

% ------------------------------------------------------------------------
%   Numeracja równań, rysunków i tabel
%   Np.: (1.2), gdzie:
%   1 - numer sekcji, 2 - numer równania, rysunku, tabeli
%   Uwaga ogólna: o otoczeniu figure ma być najpierw \caption{}, potem \label{}, inaczej odnośnik nie działa!
% ------------------------------------------------------------------------
\makeatletter
    \@addtoreset{equation}{section} %resetuje licznik po rozpoczęciu nowej sekcji
    \renewcommand{\theequation}{{\thesection}.\@arabic\c@equation} %dodaje kropkę

    \@addtoreset{figure}{section}
    \renewcommand{\thefigure}{{\thesection}.\@arabic\c@figure}

    \@addtoreset{table}{section}
    \renewcommand{\thetable}{{\thesection}.\@arabic\c@table}
\makeatother

% ------------------------------------------------------------------------
% Tablica
% ------------------------------------------------------------------------
\newenvironment{tablica}[3]
{
    \begin{table}[!tb]
    \centering
    \caption[#1]{#2}
    \vskip 9pt
    #3
}{
    \end{table}
}

% ------------------------------------------------------------------------
% Dostosowanie wyglądu pozycji listy \todos, np. zamiast 'p.' jest 'str.'
% ------------------------------------------------------------------------
\renewcommand{\todoitem}[2]{%
    \item \label{todo:\thetodo}%
    \ifx#1\todomark%
        \else\textbf{#1 }%
    \fi%
    (str.~\pageref{todopage:\thetodo})\ #2}
\renewcommand{\todoname}{Do zrobienia...}
\renewcommand{\todomark}{~uzupełnię}

% ------------------------------------------------------------------------
% Definicje
% ------------------------------------------------------------------------
\def\nonumsection#1{%
    \section*{#1}%
    \addcontentsline{toc}{section}{#1}%
    }
\def\nonumsubsection#1{%
    \subsection*{#1}%
    \addcontentsline{toc}{subsection}{#1}%
    }
\reversemarginpar %umieszcza notki po lewej stronie, czyli tam gdzie jest więcej miejsca
\def\notka#1{%
    \marginpar{\footnotesize{#1}}%
    }
\def\mathcal#1{%
    \mathscr{#1}%
    }
\newcommand{\atp}{ATP/EMTP} % Inaczej: \def\atp{ATP/EMTP}

% ------------------------------------------------------------------------
% Inne
% ------------------------------------------------------------------------
\frenchspacing                      %nie pamiętam co to jest, ale używam
%\flushbottom                       %nie pamiętam co to jest, ale nie używam
%\raggedbottom                      %nie pamiętam co to jest, ale nie używam
\interfootnotelinepenalty=10000		%nie dziel footnote na kilka stron
\hyphenation{ATP/-EMTP}             %dzielenie wyrazu w żądanym miejscu
\setlength{\parskip}{3pt}           %odstęp pomiędzy akapitami
\linespread{1.3}                    %odstęp pomiędzy liniami (interlinia)
\setcounter{tocdepth}{4}            %uwzględnianie w spisie treęci czterech poziomów sekcji
\setcounter{secnumdepth}{4}         %numerowanie do czwartego poziomu sekcji włącznie
%wygląd nagłówków
\titleformat\paragraph[hang]{\normalfont\sffamily\bfseries}{\theparagraph}{1em}{}
%\definecolor{niebieski}{rgb}{0.0,0.0,0.5}

%\pagestyle{fancy}
%% zmiana liter w zywej paginie na male
%%\renewcommand{\chaptermark}[1]{\markboth{#1}{}}
%\renewcommand{\sectionmark}[1]{\markboth{ #1}{}}
%\renewcommand{\subsectionmark}[1]{\markboth{ #1}{}}
%\renewcommand{\sectionmark}[1]{\markright{ #1}}
%%\renewcommand{\subsectionmark}[1]{\markleft{\thesubsection\ #1}}
%%\renewcommand{\subsectionmark}{} % zakomentarzuj jesli maja byc rowniez podrozdzialy
%\fancyhf{} % usun bierzace ustawienia pagin
%\fancyhead[LE,RO]{\small\bfseries\thepage}
%\fancyhead[LO]{\small\bfseries\rightmark}
%\fancyhead[RE]{\small\bfseries\leftmark}
%\renewcommand{\headrulewidth}{0.5pt}
%\renewcommand{\footrulewidth}{0pt}
%\addtolength{\headheight}{0.5pt} % pionowy odstep na kreske
%\fancypagestyle{plain}
%{%
%	\fancyhead{} % usun p. gorne na stronach pozbawionych numeracji (plain)
%	\renewcommand{\headrulewidth}{0pt} % pozioma kreska
%}
\pagestyle{fancy}
\renewcommand{\sectionmark}[1]{\markboth{#1}{}}
\renewcommand{\subsectionmark}[1]{\markright{\thesubsection\ #1}}
\fancyhf{}
\fancyhead[LE,RO]{\small\bfseries\thepage}
\fancyhead[LO]{\small\bfseries\rightmark}
\fancyhead[RE]{\small\bfseries\leftmark}
\renewcommand{\headrulewidth}{0.5pt}
\renewcommand{\footrulewidth}{0pt}
\addtolength{\headheight}{0.5pt}
\fancypagestyle{plain}
{%
 \fancyhead{}
 \renewcommand{\headrulewidth}{0pt}
}

\definecolor{code_back}{rgb}{0.9,0.9,0.9}