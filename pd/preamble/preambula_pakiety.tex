% ------------------------------------------------------------------------
% PAKIETY
% ------------------------------------------------------------------------

%różne pakiety matematyczne, warto przejrzeć dokumentację, muszą być powyżej ustawień językowych.
\usepackage{mathrsfs}   %Różne symbole matematyczne opisane w katalogu ~\doc\latex\comprehensive. Zamienia \mathcal{L} ze zwykłego L na L-transformatę.
\usepackage{eucal}      %Różne symbole matematyczne.
\usepackage{amssymb}    %Różne symbole matematyczne.
\usepackage{amsmath}    %Dodatkowe funkcje matematyczne, np. polecenie \dfac{}{} składające ułamek w trybie wystawionym (porównaj $\dfrac{1}{2}$, a $\frac{1}{2}$).

%język polski i klawiatura
\usepackage[polish]{babel}
\usepackage[OT4]{polski}
\usepackage[library/utf8]{inputenc}                       %Strona kodowa polskich znaków.

%obsługa pdf'a
\usepackage[pdftex,usenames,dvipsnames]{color}      %Obsługa kolorów. Opcje usenames i dvipsnames wprowadzają dodatkowe nazwy kolorów.
\usepackage[pdftex,pagebackref=false,draft=false,pdfpagelabels=false,colorlinks=true,urlcolor=black,linkcolor=black,citecolor=black,pdfstartview=FitH,pdfstartpage=1,pdfpagemode=UseOutlines,bookmarks=true,bookmarksopen=true,bookmarksopenlevel=2,bookmarksnumbered=true,pdfauthor={Tomasz
Flis,Dominik Kornaus},pdftitle={dokument},pdfsubject={Praca
magisterska},pdfkeywords={},unicode=true]{hyperref}
%Opcja pagebackref=true dotyczy bibliografii: pokazuje w spisie literatury
%numery stron, na których odwołano się do danej pozycji.

%bibliografia
\usepackage[numbers,sort&compress]{natbib}  %Porządkuje zawartość odnośników do literatury, np. [2-4,6]. Musi być pod pdf'em, a styl bibliogfafii musi mieć nazwę z dodatkiem 'nat', np. \bibliographystyle{unsrtnat} (w kolejności cytowania).
\usepackage{library/hypernat}                       %Potrzebna pakietowi natbib do wspołpracy z pakietem hyperref (ważna kolejność: 1. hyperref, 2. natbib, 3. hypernat).

%grafika i geometria strony
\usepackage{extsizes}           %Dostępne inne rozmiary czcionek, np. 14 w poleceniu: \documentclass[14pt]{article}.
\usepackage[final]{graphicx}
\usepackage[a4paper,left=3.5cm,right=2.5cm,top=2.5cm,bottom=2.5cm]{geometry}

%strona tytu�owa
\usepackage{titlepage}

%inne
\usepackage[hide]{library/todo}                     %Wprowadza polecenie \todo{treść}. Opcje pakietu: hide/show. Polecenie \todos ma być na końcu dokumentu, wszystkie \todo{} po \todos sa ignorowane.
%\usepackage[basic,physics]{circ}            %Wprowadza środowisko circuit do rysowania obwodów elektrycznych. Musi być poniżej pakietow językowych.
\usepackage[sf,bf,outermarks]{titlesec/titlesec}     %Troszczy się o wygląd tytułów rozdziałów (section, subsection, ...). sf oznacza czcionkę sans serif (typu arial), bf -- bold. U mnie: oddzielna linia dla nagłówku paragraph. Patrz tez: tocloft -- lepiej robi format spisu treści. \usepackage{tocloft}                        %Troszczy się o format spisu trsci. \usepackage{library/expdlist}    %Zmienia definicję środowiska description, daje większe możliwości wpływu na wygląd listy.
\usepackage{flafter}     %Wprowadza parametr [tb] do polecenia \suppressfloats[t] (polecenie to powoduje nie umieszczanie rysunków, tabel itp. na stronach, na których jest to polecenie (np. może być to strona z tytułem rozdziału, który chcemy żeby był u samej góry, a nie np. pod rysunkiem)).
\usepackage{array}       %Ładniej drukuje tabelki (np. daje wiecej miejsca w komorkach -- nie są tak ścieśnione, jak bez tego pakietu).
\usepackage{listings}    %Listingi programow.
\lstset{language=C++,basicstyle=\scriptsize}
\usepackage[format=hang,labelsep=period,font={bf,footnotesize}]{caption} %Formatuje podpisy pod rysunkami i tabelami. Parametr 'hang' powoduje wcięcie kolejnych linii podpisu na szerokosc nazwy podpisu, np. 'Rysunek 1.'. \usepackage{appendix}    %Troszczy się o załączniki. \usepackage{floatflt}    %Troszczy się o oblewanie rysunkow tekstem.
\usepackage{appendix}    %Troszczy się o załączniki.
\usepackage{floatflt}    %Troszczy się o oblewanie rysunkow tekstem.
\usepackage{here}        %Wprowadza dodtkowy parametr umiejscowienia rysunków, tabel, itp.: H (duże). Umiejscawia obiekty ruchome dokladnie tam gdzie są w kodzie źródłowym dokumentu.
\usepackage{makeidx}     %Troszczy się o indeks (skorowidz).
\usepackage{fnpos}		%Umieszcza przypisy tam gdzie trzeba
%\makeFNbottom
\usepackage{inputenc}
%nieużywane, ale potencjalnie przydatne
\usepackage{sectsty}           %Formatuje nagłówki, np. żeby były kolorowe -- polecenie: \allsectionsfont{\color{Blue}}.
\usepackage{tocloft}
\usepackage{expdlist}
%\usepackage{version}           %Wersje dokumentu.
%\usepackage{fancyhdr}          %Dodaje naglowki jakie się chce.
%\usepackage{antyktor}          %Składa dokument przy użyciu Antykwy Toruńskiej.
%\usepackage{antpolt}           %Składa dokument przy użyciu Antykwy Ptawskiego.
%\usepackage[left]{showlabels}  %Pokazuje etykiety, ale kiepsko bo nie mieszczą się na marginesie (można od biedy powiększyć margines w pakiecie geometry powyżej). Nie może być na górze (pakiet).
\usepackage{fancyhdr}		%Pagina
\usepackage{subfig}
\usepackage{indentfirst}
\usepackage{fmtcount}
\usepackage[table]{xcolor}
\usepackage{setspace}