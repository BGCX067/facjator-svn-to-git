%\section*{Abstract}
%\thispagestyle{empty}
%Grafika trójwymiarowa to jedna z najdynamiczniej rozwijających się dziedzin
%informatyki. Nieustanne dążenie do uzyskania fotorealistycznego obrazu
%wymusza dopracowywanie elementów budujących wirtualny świat oraz technik i
%programów pozwalających na uzyskanie jak najlepszych efektów w jak najkrótszym
%czasie.
%
%W tej pracy prezentujemy technikę, która pozwala na znaczne przyspieszenie
%procesu tworzenia trójwymiarowego modelu głowy dzięki wykorzystaniu wymyślonych
%przez nas metod generowania modelu ze szkiców profili głowy. Użycie
%prezentowanych metod pozwala na zautomatyzowanie części procesu odtwarzania
%obiektu i daje w efekcie podstawowe odwzorowanie analizowanego obiektu.
%Przedstawiona technika jako dane wejściowe wykorzystuje obrazy dwuwymiarowe
%zawierające rzuty głowy oraz informację o rozmieszczeniu na nich punktów
%charakterystycznych. Algorytm pozwala na uzyskanie przybliżonej reprezentacji
%trójwymiarowej głowy przy małej ilości danych, zwiększając swą dokładność wraz
%ze wzrostem danych z różnych rzutów.
%
%\newpage
\section*{Abstract}
Three-dimensional graphics is one of the fastest growing disciplines of
computer science. The pursuite of photorealistic images is a constant
process that forces developement of new algorithms and software to obtain the
best results in the shortest possible time.

In this paper we present the technique aimed to drastically decrease the time
needed to create 3D models of human head using photos of real head. Techniques
described in this paper allows creation of models to be automated and
requires only small amount of data to give rough model of analysed head.