\section{Tytuł rozdziału załącznika}

%\subsection{Tytuł podrozdziału załącznika}
%\label{sec:podrozdzial_zalacznika}

%Jakaś treść. Równanie Kirchhoffa dla obwodu przedstawionego na rys.~\ref{rys:zalacznik:rl}:

%\begin{figure}[H]
%\begin{center}
%\begin{circuit}0
%    \nl\U1 $E$ - u
%    \- 2 u
%    \- 2 r
%    \nl\cc\R1 $R$ r
%    \- 2 r
%    \nl\cc\L1 $L$ . r
%    \- 2 r
%    \- 8 d
%    \- 14 l
%    \- 2 u
%\end{circuit}
%\caption{Obwód szeregowy $RL$.}\label{rys:zalacznik:rl}
%\end{center}
%\end{figure}

%\noindent ma postać:

%\begin{equation}
%    E = u_R + u_L = Ri + L{di\over dt}.
%\end{equation}

%\noindent W zwyczajowej formie zapisuje się:

%\begin{equation}
%    \label{eq:rownanie_w_zalaczniku}
%    i(t) =  \left(i_0 - {E\over R}\right)e^{-(R/L) t} + {E\over R}.
%\end{equation}

%\subsection{Tytuł kolejnego podrozdziału załącznika}
%Jakaś treść...

%\section{Tytuł kolejnego rozdziału załącznika}
%Jakaś treść...
