\section{Rozbudowa aplikacji}
Mając do dyspozycji kod źródłowy programu można bezproblemowo dodać obsługę
nowych terminali i nieterminali. Jako przykład wykorzystana zostanie
implementacja terminala Sphere, będąca reprezentacją sfery.
Każdy nowy obiekt gramatyki musi dziedziczyć po klasie \emph{GrammarObject}
i, w zależności od tego czy jest tworzony terminal (brak elementów podrzędnych)
czy nieterminal (węzeł posiadający elementy podrzędne), implementować kilka
wybranych metod:
\begin{enumerate}
  \item nieterminal: 
  \begin{enumerate}
    \item konstruktor, może przyjmować parametry, które trzeba będzie podać w
    skrypcie gramatyki kształtu;
    \item \emph{raytrace}, która sprawdzi wynik operatora dla wszystkich
    podwęzłów;
    \item \emph{clone()}, który utworzy kopię węzła.
  \end{enumerate}
  \item terminal: 
  \begin{enumerate}
    \item konstruktor, może przyjmować parametry, które trzeba będzie podać w
    skrypcie gramatyki kształtu;
    \item \emph{hit(vl::vec4\& p, vl::vec4\& v, set\& s)}, która zwróci flagę
    oznaczającą czy element przecina promień (algorytm \emph{ray trace});
    \item \emph{hit(vl::vec4\& p)}, kóra zwróci flagę oznaczającą czy punkt
    spełnia warunek elementu np. czy znajduje się wewnątrz stożka (algorytm
    przetwarzania wokseli);
    \item \emph{clone()}, który utworzy kopię węzła.
  \end{enumerate}
\end{enumerate}

Plik nagłowkowy:
{
\small
\begin{lstlisting}[language=C++,numbers=left,frame=single,numberstyle=\tiny,backgroundcolor=\color{code_back},breaklines=true]
class Sphere : public GrammarObject { // tolua_export
    public:
    // tolua_begin
        Sphere();
        ~Sphere();
    //tolua_end
        virtual bool hit(vl::vec4& p, vl::vec4& v, set& s);
        virtual bool hit(vl::vec4& p);
        virtual Node::node_ptr clone();
}; // tolua_export
\end{lstlisting}
}
Implementacja metod:
{
\small
\begin{lstlisting}[language=C++,numbers=left,frame=single,numberstyle=\tiny,backgroundcolor=\color{code_back},breaklines=true]
bool Sphere::hit(vl::vec4& p, vl::vec4& v, set& s)
{
    //wyznacza punkty przeciecia promienia ze sfera
    //jezeli jest przeciecie to metoda zwraca true, wpp false
    vl::vec3 v_tmp = v.xyz();
    vl::vec3 p_tmp = p.xyz();
    float a = vl::dot(v_tmp, v_tmp);
    float b = vl::dot(v_tmp, p_tmp) * 2;
    float c = vl::dot(p_tmp, p_tmp) - 1;
    float d = b*b - 4*a*c;
    if (d > 0)
    {
        float sq = sqrt(d);
        float t1 = (-b-sq)/(2*a);
        float t2 = (-b+sq)/(2*a);

        vl::vec3 n1 = p_tmp + t1 * v_tmp;
        n1.normalize();
        vl::vec3 n2 = p_tmp + t2 * v_tmp;
        n2.normalize();

        if (t1 < t2)
            s.add(t1, n1, t2, n2);
        else
            s.add(t2, n2, t1, n1);
        return true;
    }
    return false;
}

bool Sphere::hit(vl::vec4& p)
{
    //zwraca true jezeli p znajduje sie wewnatrz sfery
    return p.xyz().length() < 1;
}

Node::node_ptr Sphere::clone()
{
    node_ptr n(new Sphere);
    GrammarObject::clone(n);
    return n;
}
\end{lstlisting}
}

W przykładowym kodzie zaimplementowany jest element sfery. Odpowiednie metody
\emph{hit} implementują obsługę algorytmu \emph{ray tace} oraz przetwarzania wokselowego.
Aby klasa obiektu była dostępna z poziomu gramatyki, należy zgodnie z powyższym
schematem, umieścić znaczniki rozumiane przez tolua++. Po przygotowaniu kodu
klasy należy przetworzyć plik nagłówkowy programem tolua++.exe (dla systemów
Windows). Zakładając, że plik nagłówkowy ma nazwę {\em sphere.hpp}, należy
wykonać poniższą komendę:
{
\small
\begin{lstlisting}[numbers=left,frame=single,numberstyle=\tiny,backgroundcolor=\color{code_back},breaklines=true]
tolua++ -o tolua_sphere.cpp -H tolua_sphere.hpp -n sphere sphere.hpp
\end{lstlisting}
}

Spowoduje to utworzenie plików {\em tolua\_sphere.hpp} i {\em
tolua\_sphere.cpp}, które zawierają kod udostępniający klasę {\em Sphere} do
skryptów lua. Pliki trzeba dodać do projektu i skompilować. Po takich
przygotowaniach można korzystać z utworzonej klasy w gramatyce kształtu:
{
\small
\begin{lstlisting}[numbers=left,frame=single,numberstyle=\tiny,backgroundcolor=\color{code_back},breaklines=true]
local sfera = Sphere()
\end{lstlisting}
}

W podobny sposób możemy dodać nieterminal. Jako przykład może posłużyć
modyfikator \emph{And}.

Plik nagłówkowy:

{
\small
\begin{lstlisting}[numbers=left,frame=single,numberstyle=\tiny,backgroundcolor=\color{code_back},breaklines=true]
class And : public GrammarObject { // tolua_export
    public:
        And() {}
    // tolua_begin
        And(GrammarObject& t1, GrammarObject& t2);
        And(GrammarObject& t1, GrammarObject& t2, GrammarObject& t3);
        And(GrammarObject& t1, GrammarObject& t2, GrammarObject& t3, GrammarObject& t4);
        And(GrammarObject& t1, GrammarObject& t2, GrammarObject& t3, GrammarObject& t4, 
        GrammarObject& t5);
        ~And();
    //tolua_end
        virtual bool rayTrace(vl::vec4& p, vl::vec4& v, set& s);
        virtual bool rayTrace(vl::vec4& p);
        virtual Node::node_ptr clone();
}; // tolua_export
\end{lstlisting}
}

Oraz implementacja:

{
\small
\begin{lstlisting}[numbers=left,frame=single,numberstyle=\tiny,backgroundcolor=\color{code_back},breaklines=true]
bool And::rayTrace(vl::vec4& p, vl::vec4& v, set& s)
{
    t_child_list::iterator i = childs.begin();
    bool h;
    set ss;
    // przetransformuj punkt i promien kamery
    vl::vec4 ppt = tmpMatrixInv * p;
    vl::vec4 vvt = tmpMatrixInv * v;

    // przekaz do wszystkich dzieci i sprawdz czy
    // promien przeszedl przez obiekty
    for (;i!=childs.end(); i++)
    {
        vl::vec4 pp = ppt;
        vl::vec4 vv = vvt;

        if (i == childs.begin())
        {
            h  = dynamic_cast<GrammarObject*>(*i)->rayTrace(pp, vv, ss);
        }
        else
        {
            set sss;
            h &= dynamic_cast<GrammarObject*>(*i)->rayTrace(pp, vv, sss);
            // operacja AND na zbiorze odcinkow
            ss *= sss;
        }
    }

    // zwroc true jesli zbior nie jest pusty
    if (h)
    {
        s = ss;
        return true;
    }

    return false;
}

bool And::rayTrace(vl::vec4& p)
{
    t_child_list::iterator i = childs.begin();
    bool h;
    // przemnoz punkt przez lokalna macierz
    vl::vec4 ppt = tmpMatrixInv * p;

    // przekaz nowa pozycje do wszystkich dzieci
    for (;i!=childs.end(); i++)
    {
        vl::vec4 pp = ppt;

        if (i == childs.begin())
        {
            h  = dynamic_cast<GrammarObject*>(*i)->rayTrace(pp);
        }
        else
        {
            h &= dynamic_cast<GrammarObject*>(*i)->rayTrace(pp);
        }
    }

    // zwroc true jesli punkt jest aktywny we wszystkich dzieciach
    return h;
}

Node::node_ptr And::clone()
{
    node_ptr n(new And);
    // sklonuj samego siebie
    GrammarObject::clone(n);

    return n;
}
\end{lstlisting}
}

Jak widać powyżej zaimplementowane są metody \emph{rayTrace} i metoda
\emph{clone}. Pierwsza metoda \emph{rayTrace} (używana przy renderingu metodą
ray-tracing), wylicza nową pozycję kamery i kierunek wektora promienia na
podstawie lokalnej macierzy nieterminala, a następnie przekazuje te parametry do
każdego dziecka licząc część wspólną na zbiorach odcinków (linia 26). Na końcu
jeśli zbiór odcinków nie jest pusty, zwraca \emph{true}. Analogicznie druga
metoda \emph{rayTrace} przekształca pozycję w przestrzeni wokselowej na
podstawie lokalnej macierzy i przekazuje ją do swoich dzieci, a następnie
oblicza część wspólną dla każdego z dzieci i zwraca wynik. Ostatnia metoda
\emph{clone} --- klonuje siebie samą i zwraca klona.
