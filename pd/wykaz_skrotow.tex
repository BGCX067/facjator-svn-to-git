\nonumsection{Wykaz ważniejszych skrótów i~oznaczeń}
\suppressfloats[t]

\begin{description}[\setleftmargin{65pt}\setlabelstyle{\bfseries}]
    \leftskip=1cm
    \item[derywacja]	w językoznawstwie to proces słowotwórczy polegający na
    tworzeniu wyrazów pochodnych poprzez dodawanie do podstawy słowotwórczej
    sufiksów i prefiksów
    \item[fraktal]		w znaczeniu potocznym oznacza zwykle obiekt samo-podobny
    (tzn. taki, którego części są podobne do całości) albo nieskończenie
    subtelny (ukazujący subtelne detale nawet w wielokrotnym powiększeniu)
    \item[CSG]			(ang. Constructive Solid Geometry) w grafice komputerowej i
    zastosowaniach CAD technika definiowania nowych brył poprzez łączenie innych brył
    regularyzowanymi działaniami boolowskimi: sumą, częścią wspólną i różnicą. Regularyzowane
    operatory tym różnią się od zwykłych działań na zbiorach punktów, że gwarantują,
    iż wynikiem działania będzie nadal bryła, a więc obiekt posiadający objętość. Wynikiem
    regularyzowanej operacji nie będzie zatem nigdy punkt, odcinek lub płaszczyzna.
    \item[ray-tracing]	śledzenie promieni --- technika
    generowania fotorealistycznych obrazów scen trójwymiarowych. Opiera się na analizowaniu tylko
    tych promieni światła, które trafiają bezpośrednio do obserwatora. W rekursywnym
    śledzeniu promieni bada się dodatkowo promienie odbite zwierciadlane oraz załamane.
    Ponadto umożliwia łatwą realizację CSG, a także wizualizację idealnych,
    opisywanych formułami matematycznymi obiektów.
%    \item[PE]			Portable Executable
%    \item[$f$]        częstotliwość źródeł sieci elektroenergetycznej
%    \item[$\omega$]   pulsacja źródeł sieci elektroenergetycznej
%    \item[$\omega_0$] pulsacja drgań własnych obwodu drgającego
%    \item[$U_n$]      napięcie znamionowe sieci elektroenergetycznej, wartość skuteczna napięcia międzyfazowego sieci
%    \item[$E$]        wartość skuteczna fazowego napięcia znamionowego sieci
\end{description}
