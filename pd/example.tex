%Równanie:
%\begin{equation}
%    \label{eq:rownanie}
%    i(t) =  \left(i_0 - {E\over R}\right)e^{-(R/L) t} + {E\over R}.
%\end{equation}

%\begin{figure}[!tb]
%    \centering
%    \includegraphics[width=9.82cm]{graphics/rysunek.pdf}\\
%    \caption
%    %uwaga: w nawiasach [] nie może być odnośnika do literatury, jeśli w dokumencie jest spis rysunków na początku, a spis literatury jest w kolejności cytowania (zmienia to numerację)
%    [Przykładowy rysunek z Excela. Długi tytuł rysunku, żeby pokazać ułożenie tytułu na stronie.]
%    {Przykładowy rysunek z Excela~\cite{bronsztejn}. Długi tytuł rysunku, żeby pokazać ułożenie tytułu na
%stronie.}
%    \label{fig:rysunek}
%\end{figure}


%\begin{figure}[!tb]
%    \begin{center}
%    \input{graphics/wykres.tex}
%    \caption{Przykładowy wykres z Gnuplota.}
%    \label{rys:przykladowy_gnuplot}
%    \end{center}
%\end{figure}


%\begin{figure}[!tb]
%\begin{center}
%\begin{circuit}0
%    \nl\U1 $E$ - u
%    \- 2 u
%    \- 2 r
%    \nl\cc\R1 $R$ r
%    \- 2 r
%    \nl\cc\L1 $L$ . r
%    \- 2 r
%    \- 8 d
%    \- 14 l
%    \- 2 u
%\end{circuit}
%\caption{Przykładowy obwód elektryczny (pakiet circ.sty).}\label{rys:przykladowy_obwod}
%\end{center}
%\end{figure}

%ustawienia tablicy
%\setlength{\tabcolsep}{1.0em}
%\renewcommand{\arraystretch}{1.5}

%\begin{tablica}
    %uwaga: w nawiasach [] nie może być odnośnika do literatury, jeśli w dokumencie jest spis rysunków na początku, a spis literatury jest w kolejności cytowania (zmienia to numerację)
%    {Przykładowa tablica.}
%    {Przykładowa tablica~\cite{slepian}.}
%    {
%    \begin{tabular}{|c|c|} \hline
%        $U_n$ & $I_{zw}$ \\ \hline
%        $kV$  & $\%$      \\ \hline
%        7.2 & 100 \\ \hline
%    \end{tabular}
%    }
%    \label{tab:przykladowa_tablica}
%\end{tablica}
