\section{Krótkie wprowadzenie do Lua}
Rozdział zawiera wprowadzenie do programowania w języku Lua, który jest
wykorzystywany do tworzenia gramatyki kształtu w programie \emph{Facjator}.
Opisane są tylko najbardziej elementarne elementy języka, więcej informacji
można uzyskać z dokumentacji\footnote{http://www.lua.org/docs.html}.

Do przechowywania tymczasowych wartości służą w Lua zmienne, które mogą mieć
zasięg globalny lub lokalny. Zmienne nie mają przypisanego typu, a typ powiązany
ze zmienną jest determinowany przez aktualnie przechowywaną wartość.
Dopuszczalne wartości mogą reprezentować:
\begin{enumerate}
  \item \emph{nil} czyli brak wartości, pusta zmienna;
  \item liczby całkowite, zmiennoprzecinkowe;
  \item ciągi znaków (\emph{stringi});
  \item wartość logiczną \emph{true} lub \emph{false};
  \item funkcje;
  \item tablice.
\end{enumerate}
Definiowanie nowej zmiennej polega na zwykłym przypisaniu wartości,
jeżeli zmienna o podanej nazwie już istniała to zostaje nadpisana. Do
definiowania zmiennej lokalnej używa się słowa kluczowego \emph{local}. Nazwa
zmiennej musi zaczynać się od litery lub podkreślenia, dalsza część może składać
się jedynie z liter, cyfr oraz podkreślenia, jako nazw nie można używać słów
kluczowych. Przy korzystaniu ze zmiennych należy pamiętać, że wielkość liter ma
znaczenie.
{
\small
\begin{lstlisting}[numbers=left,frame=single,numberstyle=\tiny,backgroundcolor=\color{code_back},breaklines=true]
zmienna = 5
moj_string = "Praca dyplomowa"
local zmienna_lokalna = true
\end{lstlisting}
}

Do warunkowego wykonywania części gramatyki można wykorzystać instrukcję
warunkową \emph{if}. Instrukcja sprawdza warunek i jeżeli jest on spełniony to
wykonuje się ujęty nią fagment kodu. Można wykorzystać wersję uwzględniającą
warunek \emph{else}.
{
\small
\begin{lstlisting}[numbers=left,frame=single,numberstyle=\tiny,backgroundcolor=\color{code_back},breaklines=true]
if zmienna < 5 then
  [...]
elseif zmienna == 5 then
  [...]
else
  [...]
end
\end{lstlisting}
}
Warunki można sprawdzać operatorami: równości (==), nierówności (~= lub !=),
mniejszości (<), większości (>), mniejszy lub równy (<=), większy lub równy
(>=).

Do wielokrotnego wykonywania tego samego fragmentu kodu można wykorzystać pętle,
które mogą przyjmować jedną z poniższych form:
{
\small
\begin{lstlisting}[numbers=left,frame=single,numberstyle=\tiny,backgroundcolor=\color{code_back},breaklines=true]
for nazwa_zmiennej = poczatek,koniec do instrukcje end
\end{lstlisting}
}
{
\small
\begin{lstlisting}[numbers=left,frame=single,numberstyle=\tiny,backgroundcolor=\color{code_back},breaklines=true]
repeat instrukcje until warunek
\end{lstlisting}
}
{
\small
\begin{lstlisting}[numbers=left,frame=single,numberstyle=\tiny,backgroundcolor=\color{code_back},breaklines=true]
while warunek do instrukcje end
\end{lstlisting}
}


Podczas tworzenia rozbudowanego kodu przydatne może okazać się grupowanie jego
części w logicznie fragmenty. Do tego celu wykorzystać można funkcje.
Definicja funkcji składa się ze słowa kluczowego \emph{function}, nazwy, listy
parametrów ujętych w nawiasy, treści funkcji oraz słowa kluczowego \emph{end}.
Jeżeli funkcja ma zwracać wartość to należy użyć słowa kluczowego \emph{return}.
W funkcji można definiować zmienne lokalne oznaczając ją jako \emph{local}.

{
\small
\begin{lstlisting}[numbers=left,frame=single,numberstyle=\tiny,backgroundcolor=\color{code_back},breaklines=true]
function toCelsius(input)
  -- zmienna lokalna, niewidoczna poza funkcja
  local ret = (input - 32) * (5 / 9) 
  return ret --zwraamy wartosc
end

-- wywolanie funkcji
tempCelsius = toCelsius(45)
\end{lstlisting}
}